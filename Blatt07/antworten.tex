\documentclass[a4paper,10pt]{article}
\usepackage[utf8]{inputenc}
\usepackage{amsmath}

%opening
\title{Aufgabe 07}
\author{Can Nayci, Leonhard Rattmann, Emil Sharaf}
\date{09.12.2023}
\begin{document}

\maketitle

\section{Circle}
\begin{align*}
 N &: \text{Anzahl der Elemente des Arrays} \\
 n &: \text{Index des Elements} \\
 nprocs &: \text{Anzahl der Prozesse} \\
 start_p &: \text{erstes Element für Prozess } p\\
 end_p &: \text{letztes Element für Prozess } p\\
 \\
 start_p(n) &= (int) n \cdot \frac{N}{nprocs}\\
 end_p(n) &= (int) n \cdot \frac{N}{nprocs}
\end{align*}\\
für $N = 13, nprocs = 5$:
\begin{tabular}{c | c}
 Prozess & Array-Elemente \\
 \hline
 0 & $0,1,2$\\
 1 & $3,4,5$\\
 2 & $6,7$\\
 3 & $8,9,10$\\
 4 & $11,12$\\
\end{tabular}

\end{document}
