\documentclass[a4paper,10pt]{article}
\usepackage[utf8]{inputenc}

%opening
\title{Blatt 09}
\author{Nayci, Rattmann, Sharaf}
\date{23.12.2023}
\begin{document}

\maketitle

\section{}
$t_0$:
\begin{enumerate}
 \item $p_0$ berechnet seinen Block.
 \item $p_0$ sendet die Haloline (= unterste Zeile) an $p_1$.
 \item $p_0$ sendet sein MaxResiduum an $p_1$.
\end{enumerate}
$t_1-t_5:$
\begin{itemize}
\item $\forall p_x \setminus p_0$:
\subitem für Iteration $i = 1$: Empfange Haloline von $p_{x-1}$
\subitem für $i \neq 1$: Empfange Haloline von $p_{x-1}$ und $p_{x+1}$
\item $\forall p_x$:
\item Berechne Block.
\item $\forall p_x \setminus \{p_0, p_{n-1}\}$:
\subitem Sende relevante Halolines an $p_{x-1}$ und $p_{x+1}$
\item für $p_0$: Sende untere Haloline an $p_1$
\item für $p_{n-1}$: Sende obere Haloline an $p_{n-2}$
\item $\forall p_x \setminus p_{n-1}$: Sende MaxResiduum an $p_{x+1}$.
\item für $p_{n-1}$: (Wenn TERM\_PREC): Überprüfe ob Genauigkeits-Abbruch. Wenn ja, sende kollektive Flag. Sende Abbruchiteration ($i(p_{n-1}) + n - 1$). In diesem Fall $i(p_{n-1}) = 3$.
\end{itemize}
$t_6:$
\begin{itemize}
 \item $\forall p_x \setminus p_0:$ Kommuniziere Matrixwerte an $p_0$.
\end{itemize}

\begin{tabular}{c | c c c c c c c}
 Prozess & t0 & t1 & t2 & t3 & t4 & t5 & t6\\
 \hline
 p0 & 1 & 2 & 3 & 4 & 5 & - & - \\
 p1 & - & 1 & 2 & 3 & 4 & 5 & - \\
 p2 & - & - & 1 & 2 & 3 & 4 & 5 \\
\end{tabular}


\end{document}
